%
% titlepage.tex
%
% Copyright (C) 2023 UFSC.
%
% DOCUMENTATION-TEMPLATE
%
% This work is licensed under the Creative Commons Attribution-ShareAlike 4.0
% International License. To view a copy of this license,
% visit http://creativecommons.org/licenses/by-sa/4.0/.
%

\begin{titlepage}

\thispagestyle{empty}

\begin{flushleft}
\end{flushleft}

\vspace{1cm}

\begin{figure}[!ht]
    \begin{flushleft}
        \includegraphics[width=7cm]{figures/horizontal_fundo_claro.png}
    \end{flushleft}
\end{figure}

\begin{flushleft}
\Huge{\textbf{\thetitle}}
\rule[0pt]{\textwidth}{5pt}
\end{flushleft}

\vspace{0.2cm}

\begin{flushleft}
\textit{\thetitle} \\
\textit{Universidade Federal de Santa Catarina, Florianópolis - Brazil}
\end{flushleft}

\vfill
\vfill

\begin{flushright}
\monthyeardate\today
\end{flushright}

\end{titlepage}
